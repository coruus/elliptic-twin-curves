\documentclass[11pt,fleqn]{article}
\usepackage{fontspec}
\defaultfontfeatures{ Scale = MatchLowercase }
\defaultfontfeatures[\rmfamily]{ Scale = 1 }
\fontspec{XITS}[Numbers=SlashedZero,Mapping=tex-text]

\usepackage{amsmath}
\usepackage{mathtools}

\usepackage[math-style=ISO,bold-style=ISO,vargreek-shape=unicode]{unicode-math}
\setmathfont{XITSMath}
\setmathfont[range={\mathcal,\mathbfcal},StylisticSet=1]{XITSMath}

\usepackage{tabularx}
\usepackage[psdextra=true,pdfversion=1.7,pdfusetitle=true]{hyperref}

\hypersetup{breaklinks=true,
            bookmarks=true,
            colorlinks=true,
            urlcolor=blue,
            linkcolor=blue,
            pdfborder={0 0 0}}
            \usepackage[open,depth=6]{bookmark}
\usepackage{indentfirst}
\usepackage[backend=biber,style=authoryear,natbib=true,mcite=true,hyperref=true,arxiv=abs]{biblatex}
\addbibresource{soleil.bib}

\title{Distribution of $\#E^t$ for prime-order elliptic curves: Numerical results}
\author{David Leon Gil}

\newcommand{\Z}{\ensuremath{\BbbZ_q} }
\newcommand{\Zq}{\ensuremath{\BbbZ_q} }
\newcommand{\Ejt}{\ensuremath{\BbbE^t_j(\BbbZ_q))} }
\newcommand{\Tf}{\ensuremath{T_f(j)} }
\newcommand{\Pa}{\ensuremath{\mathup{P}_{224}} }
\newcommand{\Pb}{\ensuremath{\mathup{P}_{256}} }
\newcommand{\Pc}{\ensuremath{\mathup{P}_{384}} }
\newcommand{\Pd}{\ensuremath{\mathup{M}_{255}} }
\newcommand{\Pg}{\ensuremath{\mathup{H}_{448}} }

\newcommand{\Ej}{\ensuremath{\BbbE_j} }
\newcommand{\Ejfq}{\ensuremath{\BbbE_j(\BbbF_q)} }
\newcommand{\Ejfqt}{\ensuremath{\widetilde{\BbbE}_j(\BbbF_q)} }
\newcommand{\Ejfp}{\ensuremath{\BbbE_j(\BbbF_p)} }
\newcommand{\Ejfpt}{\ensuremath{\widetilde{\BbbE}_j(\BbbF_p)} }


\begin{document}

\maketitle

\section{Preface}

This is a preliminary note of some numerical experiments; the results
may be rather wrong.

\section{Introduction}

Twist-secure curves for elliptic curve cryptography are much in vogue
these days; \cite{djb} and others have proposed that twist-security
is an essential safety condition for choosing curves.

\cite{curve25519} cites \cite{KaliskiJCryptology} as introducing the
so-called ``unsafe-twist'' attack,f but I have been unable to find any
evidence either there or in his (quite excellent) thesis, \cite{KaliskiThesis},
that he was aware of this attack.

\section{Elliptic twin curves}

We follow the definitions of \autocite{ShparlinskiSutantyo}, with some minor
modifications.

Let $\Ej$ be the elliptic curve of invariant $j$, and $\Ejfq$ be its reduction
over a finite field of characteristic $p > 5, n \geq 1$ with $p$ prime. Let
and $t(\Ejfq)$ be the trace of Frobenius of that elliptic curve.

Let $\Ejfqt$ be the non-trivial quadratic twist of $\Ejfq$ over the same field.

An \emph{elliptic twin} is a pair consisting of a prime $p$, and two primes
not equal to $p$ or $0$, $(p, {l, r})$ such that

\begin{equation}
\#\Ejfp \#\Ejfpt = l + r = 2 p + 2 - l + r
\end{equation}

It is clear that there exist elliptic twins over arbitrary prime fields, but
the formulae of \autocite{ShparlinskiSutantyo} are not amenable to characterizing
local fluctuations in the density of such elliptic twins holding the finite
field fixed.

\section{Primes}

We consider the non-Mersenne SECP primes, standardized for the use of the
federal government in \cite{recur}, which are, where $N \coloneqq 2^{32}$:

\begin{equation}
\begin{aligned}
    \Pa &= N^7 - N^3 + N^0                \\
    \Pb &= N^8 - N^7 + N^6 + N^3 - N^0    \\
    \Pc &= N^{12} - N^4 - N^3 + N^1 - N^0
\end{aligned}
\end{equation}

They are subset of the class of Generalized Mersennes defined by
\autocite{Solinas}.

%\autocite{grangermoss},

In future work, we plan to extend the study to consider the more general question
of the distribution of group structure and curve exponent for the reduction of
curves over fields for which the number of points is non-prime, and apply similar
techniques with respect to the two curves proposed for IETF use, the nearly-Mersenne
$\Pd = 2^{255}-19$ and the Hamburg-Solinas trinomial $\Pg = 2^{448}-2^{224}-1$.
\footnote{The Hamburg primes are "Karatsuba-friendly" \cite{Hamburg}, who was the
first to realize that these are extremely efficient for elliptic curve cryptography.}

(We probably won't extend this work to the Mersenne $\textup{M}_{521}$, as that
particular calculation is pestiferous.)

\section{Numerical methods}

\subsection{Finding prime-order curves}

A slightly modified version of PARI/GP was used to calculate the
traces of prime-order curves, based on code of \cite{Hamburg}.
(The particular code used for this version of this paper may be
found at \autocite{junkpari}.) Point-counting was aborted early
if $\#\Ej$ was found to have a small prime factor.

\subsection{Range} 

We calculate \Tf for each \Ej for $0 < j < 2^{20}, j \neq 1728$, then
test $\#\Ej$ and $\#\Ejt$ for (pseudo-)primality.

(For this to be a reasonable procedure, it requires the assumption that
j-invariant is not correlated with the probability of the curve being
an elliptic twin, even locally on the scale of 2^20.)

\subsection{Results} 

\begin{tabular}[l]{l|lll}
      & $N_{\pi}$ & $N_{\pi'}$ & $N_{\pi'} / N_{\pi}$ \\
  \Pa & 2790 & 31 & 1.1e-2 \\
  \Pb & 1956 & 15 & 0.8e-2 \\
  \Pc & 1131 & 20 & 1.8e-2 \\
\end{tabular}

\section{Future work}

Because the above method results in fairly low precision in estimating
$N_{\pi'} / N_{\pi}$ because of the small number of doubly-prime curves,
we plan to use two slightly different methods, essentially similar to those
of \autocite{ShparlinskiSutantyo}.

\emph{Procedure 1.} We generate $N$ random j-invariant, $(j_{0,0}, ..., $j_{N,0})$,
and increment each by 1 until we find a doubly prime curve.

\emph{Procedure 2.} We generate random $j$-invariants repeatedly until we collect
N primes.

The procedure we will use to generate the random number is, for procedure one,

\begin{code}
for i in range(N):
  ok = False
  while not ok:
    input = shake256("Elliptic Twins over GF(0x%x), procedure 1: i=%u, try=%u".format(P, j, try))
    maybep = l2b(shake256.squeeze(bitlen / 8))
    if maybep < P:
      ok = True
  js[i] = maybep
\end{code}

and similarly for procedure 2.

\section{Concluding, mostly irrelevant aside}

The quantity $1 / p_q = N_{\pi}(p) / N_{\pi'}(p)$ is an estimator for
the number of trials required, when choosing a prime curve uniformly at
random in $\BbbF_q$ for that curve to be an elliptic twin.

The probability, however, that no elliptic curves in a set of $N$
are elliptic twins is, of course,

\begin{equation}
1 - \Prod_{0 \leq i < n} p_q
\end{equation}

With respect to the curves standardized by NIST and generated by the
NSA, this calculation gives a probability of very approximately $< 5\%$
of any of the curves over $\Pa, \Pb, \Pc$ being elliptic twins.

A Bayesian might thus conclude that it is more likely than not that
the NSA's curves were not generated by a process that samples from
a uniform distribution on prime-order curves over the chosen prime fields.

An appropriate prior might include both

\begin{itemize}

\item The fact that the NSA has specified only two of those curves for
military use as part of its Suite B, and -- based on publicly disclosed
information -- uses P-384 or classified cryptography primarily for its
own use.

\item The previously disclosed evidence, w.r.t. DES, that the NSA has
additional safety criteria for cryptography which it does not disclose
in standardization processes.

\end{itemize}

All this suggests that the NSA's choice was subject to additional, not
publicly disclosed, safety criteria.

\section{Acknowledgments}

The patch to PARI/GP used is derived from a patch by Michael Hamburg.

All of the non-trivial mathematics is entirely derived from prior work by
Robert Ransom. The only novelty is in the application to this particular
question.

\section*{Appendix. Cofactors for SafeCurves}

This table is adapted (read stolen directly) from \cite{safecurves}.

% TODO The columns have been sorted by the ratio of the exponent of the twist to the exponent of the curve

\newcolumntype{R}{>{\raggedright\arraybackslash}X}%
%\begin{table}[h]
\begin{tabularx}{\textwidth}{llR}
\textbf{Curve}            &$h(\Ej)$ & $h(\Ejt)$ \\
\hline
\texttt{\footnotesize M221         }& $\scriptstyle 2^3$    & $\scriptstyle 2^2                                                                  $\\
\texttt{\footnotesize E222         }& $\scriptstyle 2^2$    & $\scriptstyle \mathbf{2^2}                                                                  $\\
\texttt{\footnotesize secp224r1     }& $\scriptstyle 1  $    & $\scriptstyle 3^2 \cdot 11 \cdot 47 \cdot 3015283 \cdot 40375823 \cdot 267983539294927                 $\\
\texttt{\footnotesize Curve1174     }& $\scriptstyle 2^2$    & $\scriptstyle \mathbf{2^2}                                                                  $\\
\texttt{\footnotesize Curve25519    }& $\scriptstyle 2^3$    & $\scriptstyle 2^2                                                                  $\\
\texttt{\footnotesize BN(2,254)     }& $\scriptstyle 1  $    & $\scriptstyle 3^3 \cdot 3583 \cdot 298908837206431 \cdot 11711184643015782903697616449         $\\
\texttt{\footnotesize brainpoolP256 }& $\scriptstyle 1  $    & $\scriptstyle 5^2 \cdot 175939 \cdot 492167257 \cdot 8062915307 \cdot 2590895598527 \cdot 4233394996199$\\
\texttt{\footnotesize FRP256v1      }& $\scriptstyle 1  $    & $\scriptstyle 7 \cdot 439 \cdot 11760675247 \cdot 3617872258517821                             $\\
\texttt{\footnotesize secp256r1     }& $\scriptstyle 1  $    & $\scriptstyle 3 \cdot 5 \cdot 13 \cdot 179                                                     $\\
\texttt{\footnotesize secp256k1     }& $\scriptstyle 1  $    & $\scriptstyle 3^2 \cdot 13^2 \cdot 3319 \cdot 22639                                            $\\
\texttt{\footnotesize E382          }& $\scriptstyle 2^2$    & $\scriptstyle \mathbf{2^2}                                                                  $\\
\texttt{\footnotesize M383          }& $\scriptstyle 2^3$    & $\scriptstyle 2^2                                                                  $\\
\texttt{\footnotesize Curve383187   }& $\scriptstyle 2^3$    & $\scriptstyle 2^2                                                                  $\\
\texttt{\footnotesize brainpoolP384 }& $\scriptstyle 1  $    & $\scriptstyle 7 \cdot 11^2 \cdot 241 \cdot 5557 \cdot 125972502705620325124785968921221517         $\\
\texttt{\footnotesize secp384r1     }& $\scriptstyle 1  $    & $\scriptstyle \mathbf{1}                                                                    $\\
\texttt{\footnotesize Curve41417    }& $\scriptstyle 2^3$    & $\scriptstyle \mathbf{2^3}                                                                  $\\
\texttt{\footnotesize Ed448         }& $\scriptstyle 2^2$    & $\scriptstyle \mathbf{2^2}                                                                  $\\
\texttt{\footnotesize M511          }& $\scriptstyle 2^3$    & $\scriptstyle 2^2                                                                  $\\
\texttt{\footnotesize E521          }& $\scriptstyle 2^2$    & $\scriptstyle \mathbf{2^2}$ \\
\hline
\end{tabularx}
%\end{table}


\printbibliography

\end{document}
