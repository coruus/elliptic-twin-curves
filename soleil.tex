\documentclass[11pt,fleqn]{article}
\usepackage{amsmath}
\usepackage{mathtools}
\usepackage{unicode-math}


\title{Distribution of $\#E^t$ for prime-order elliptic curves: Numerical results}
\author{David Leon Gil}

\newcommand{\Z}{\ensuremath{\BbbZ_q}}
\newcommand{\Zq}{\ensuremath{\BbbZ_q}}
\newcommand{\Ej}{\ensuremath{E(\BbbZ_q)_j}}
\newcommand{\Ejt}{\ensuremath{E^t(\BbbZ_j)}}
\newcommand{\Tf}{\ensuremath{T_f(j)}}
\newcommand{\Pa}{\ensuremath{P_{224}}}
\newcommand{\Pb}{\ensuremath{P_{256}}}
\newcommand{\Pc}{\ensuremath{P_{384}}}

\begin{document}

\section{Introduction}

This is a preliminary note of some numerical experiments; the results
may be rather wrong.

\section{Prime-order curves}

Let $\Zq$ be a finite field in characteristic greater than 3,
$\Ej$ be the elliptic curve defined by

\begin{equation}
\Ej \equiv x^3 = y^2 + \frac{36 x}{j - 1728} - \frac{1}{j - 1728}
\end{equation}

and $\Ejt$ be some curve automorphic to its quadratic twist

\begin{equation}
\Ejt \equiv x^3 = y^2 + \beta^2 \frac{36 x}{j - 1728} - \beta^3 \frac{1}{j - 1728}
\end{equation}

where $\beta$ is a quadratic non-residue in \Zq.

Then

\begin{equation}
\begin{aligned}
\#\Ej  =& q - \Tf + 1 \\
\#\Ejt =& q + \Tf + 1 \\
\end{aligned}
\end{equation}

for some \Tf, which is the trace of \Ej.

We observe that the mapping between the $j$-invariant and $c$, as
defined in XXXX, is an isomorphism.

\section{Numerical methods}

\subsection{Finding prime-order curves}

A slightly modified version of PARI/GP was used to calculate the
traces of prime-order curves. Point-counting was aborted early
if $\#K$ was found to have a small prime factor.

\subsection{Range} 

We calculate \Tf for each \Ej for $0 < j < 2^20, j != 1728$.

\section{Results} 


\begin{tabular}[l]{l|rrrr}
  $q$ & $N_{\pi}$ & $N_{\pi'}$ & $N_{\pi'} / N_{\pi}$ \\
  \Pa & \\
  \Pb & \\
  \Pc & \\
  
\end{tabular}

%http://arxiv.org/pdf/0902.4332.pdf 

\end{document}
