\documentclass[11pt,fleqn]{article}
\usepackage{fontspec}
\defaultfontfeatures{ Scale = MatchLowercase }
\defaultfontfeatures[\rmfamily]{ Scale = 1 }
\fontspec{XITS}[Numbers=SlashedZero,Mapping=tex-text]

\usepackage{amsmath}
\usepackage{mathtools}

\usepackage[math-style=ISO,bold-style=ISO,vargreek-shape=unicode]{unicode-math}
\setmathfont{XITSMath}
\setmathfont[range={\mathcal,\mathbfcal},StylisticSet=1]{XITSMath}

\usepackage{tabularx}
\usepackage[psdextra=true,pdfversion=1.7,pdfusetitle=true]{hyperref}

\hypersetup{breaklinks=true,
            bookmarks=true,
            colorlinks=true,
            urlcolor=blue,
            linkcolor=blue,
            pdfborder={0 0 0}}
            \usepackage[open,depth=6]{bookmark}
\usepackage{indentfirst}
\usepackage[backend=biber,style=authoryear,natbib=true,mcite=true,hyperref=true,arxiv=abs]{biblatex}
\addbibresource{soleil.bib}

\title{Distribution of $\#E^t$ for prime-order elliptic curves: Numerical results}
\author{David Leon Gil}


\newcommand{\Z}{\ensuremath{\BbbZ_q} }
\newcommand{\Zq}{\ensuremath{\BbbZ_q} }
\newcommand{\Ej}{\ensuremath{\BbbE_j(\BbbZ_q)} }
\newcommand{\Ejt}{\ensuremath{\BbbE^t_j(\BbbZ_q))} }
\newcommand{\Tf}{\ensuremath{T_f(j)} }
\newcommand{\Pa}{\ensuremath{\mathup{P}_{224}} }
\newcommand{\Pb}{\ensuremath{\mathup{P}_{256}} }
\newcommand{\Pc}{\ensuremath{\mathup{P}_{384}} }
\newcommand{\Pd}{\ensuremath{\mathup{M}_{255}} }
\newcommand{\Pg}{\ensuremath{\mathup{H}_{448}} }

\begin{document}

\maketitle

\section{Introduction}

This is a preliminary note of some numerical experiments; the results
may be rather wrong.

\section{Prime-order curves}

Let $\Zq$ be a finite field in characteristic greater than 3, and
$\Ej$ be the elliptic curve defined by

\begin{equation}
\Ej \equiv x^3 = y^2 + \frac{36 x}{j - 1728} - \frac{1}{j - 1728}
\end{equation}

and $\Ejt$ be some curve automorphic to its quadratic twist

\begin{equation}
\Ejt \equiv x^3 = y^2 + \beta^2 \frac{36 x}{j - 1728} - \beta^3 \frac{1}{j - 1728}
\end{equation}

where $\beta$ is a quadratic non-residue in \Zq.

Then the trace, \Tf, of \Ej, is defined by:

\begin{equation}
\begin{aligned}
&\#\Ej  &= q - \Tf + 1 \\
&\#\Ejt &= q + \Tf + 1 \\
\end{aligned}
\end{equation}

\section{Primes}

We consider the SECP primes, first suggested by \autocite{Solinas}, which are,
where $N \coloneqq 2^{32}$:

\begin{equation}
\begin{aligned}
    \Pa &= N^7 - N^3 + N^0                \\
    \Pb &= N^8 - N^7 + N^6 + N^3 - N^0    \\
    \Pc &= N^{12} - N^4 - N^3 + N^1 - N^0
\end{aligned}
\end{equation}

as well as the two curves proposed for IETF use, the nearly-Mersenne $\Pd = 2^{255}-19$
and the Goldilocks prime $\Pg = 2^{448}-2^{224}-1$.

\section{Numerical methods}

\subsection{Finding prime-order curves}

A slightly modified version of PARI/GP was used to calculate the
traces of prime-order curves. Point-counting was aborted early
if $\#K$ was found to have a small prime factor.

\subsection{Range} 

We calculate \Tf for each \Ej for $0 < j < 2^{20}, j \neq 1728$, then
test $\#\Ej$ and $\#\Ejt$ for (pseudo-)primality.

\subsection{Results} 


\begin{tabular}[l]{l|lll}
      & $N_{\pi}$ & $N_{\pi'}$ & $N_{\pi'} / N_{\pi}$ \\
  \Pa & 2790 & 31 & 1.11e-2 \\
  \Pb & 1956 & 15 & 0.77e-2 \\
  \Pc & 1131 & 20 & 1.77e-2 \\
\end{tabular}

\section{Stopping times for prime and doubly-prime curves}

Because the above method results in fairly low precision in estimating
$N_{\pi'} / N_{\pi}$ because of the small number of doubly-prime curves,
we use a slightly different method, essentially similar to that of \autocite{shpar2014elltwin}.

We generate a random j-invariant, $j_{0,0}$, and increment it by 1 until
we find a doubly prime curve (using the early-stop technique both for the
curve and its twist).

We can then roughly approximate the probability of finding a doubly-prime
curve over each field as \dots

\section{Distribution of factors for the twists of prime curves}

We consider the following question: 

\section{Acknowledgments}

The patch to PARI/GP used is derived from a patch by Michael Hamburg.

\section*{Appendix. Cofactors for SafeCurves}


Define $h(\Ej)$ as $\lbrace p^n : p^n \mid \#\Ej \rbrace \setminus \lbrace p^n : p^n \mid \#\Ejt,
\nexists p' < p\rbrace$.

This table is adapted (read stolen directly) from \cite{safecurves}.


\newcolumntype{R}{>{\raggedright\arraybackslash}X}%
%\begin{table}[h]
\begin{tabularx}{\textwidth}{llR}
\textbf{Curve}            &$h(\Ej)$ & $h(\Ejt)$ \\
\hline
\texttt{\footnotesize M221         }& $\scriptstyle 2^3$    & $\scriptstyle 2^2                                                                  $\\
\texttt{\footnotesize E222         }& $\scriptstyle 2^2$    & $\scriptstyle \mathbf{2^2}                                                                  $\\
\texttt{\footnotesize secp224r1     }& $\scriptstyle 1  $    & $\scriptstyle 3^2 \cdot 11 \cdot 47 \cdot 3015283 \cdot 40375823 \cdot 267983539294927                 $\\
\texttt{\footnotesize Curve1174     }& $\scriptstyle 2^2$    & $\scriptstyle \mathbf{2^2}                                                                  $\\
\texttt{\footnotesize Curve25519    }& $\scriptstyle 2^3$    & $\scriptstyle 2^2                                                                  $\\
\texttt{\footnotesize BN(2,254)     }& $\scriptstyle 1  $    & $\scriptstyle 3^3 \cdot 3583 \cdot 298908837206431 \cdot 11711184643015782903697616449         $\\
\texttt{\footnotesize brainpoolP256 }& $\scriptstyle 1  $    & $\scriptstyle 5^2 \cdot 175939 \cdot 492167257 \cdot 8062915307 \cdot 2590895598527 \cdot 4233394996199$\\
\texttt{\footnotesize FRP256v1      }& $\scriptstyle 1  $    & $\scriptstyle 7 \cdot 439 \cdot 11760675247 \cdot 3617872258517821                             $\\
\texttt{\footnotesize secp256r1     }& $\scriptstyle 1  $    & $\scriptstyle 3 \cdot 5 \cdot 13 \cdot 179                                                     $\\
\texttt{\footnotesize secp256k1     }& $\scriptstyle 1  $    & $\scriptstyle 3^2 \cdot 13^2 \cdot 3319 \cdot 22639                                            $\\
\texttt{\footnotesize E382          }& $\scriptstyle 2^2$    & $\scriptstyle \mathbf{2^2}                                                                  $\\
\texttt{\footnotesize M383          }& $\scriptstyle 2^3$    & $\scriptstyle 2^2                                                                  $\\
\texttt{\footnotesize Curve383187   }& $\scriptstyle 2^3$    & $\scriptstyle 2^2                                                                  $\\
\texttt{\footnotesize brainpoolP384 }& $\scriptstyle 1  $    & $\scriptstyle 7 \cdot 11^2 \cdot 241 \cdot 5557 \cdot 125972502705620325124785968921221517         $\\
\texttt{\footnotesize secp384r1     }& $\scriptstyle 1  $    & $\scriptstyle \mathbf{1}                                                                    $\\
\texttt{\footnotesize Curve41417    }& $\scriptstyle 2^3$    & $\scriptstyle \mathbf{2^3}                                                                  $\\
\texttt{\footnotesize Ed448         }& $\scriptstyle 2^2$    & $\scriptstyle \mathbf{2^2}                                                                  $\\
\texttt{\footnotesize M511          }& $\scriptstyle 2^3$    & $\scriptstyle 2^2                                                                  $\\
\texttt{\footnotesize E521          }& $\scriptstyle 2^2$    & $\scriptstyle \mathbf{2^2}$ \\
\hline
\end{tabularx}
%\end{table}


\printbibliography

%http://arxiv.org/pdf/0902.4332.pdf

\end{document}
